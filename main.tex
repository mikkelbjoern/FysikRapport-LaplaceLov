\documentclass[10pt,a4paper]{article}
\usepackage[utf8]{inputenc}
\usepackage[danish]{babel}
\usepackage{amsmath}
\usepackage{amsfonts}
\usepackage{amssymb}
\usepackage{graphicx}
\usepackage[left=2cm,right=2cm,top=2cm,bottom=2cm]{geometry}


\usepackage{titlepic}
\usepackage{enumerate}
\usepackage{enumitem}
\usepackage{float}
\usepackage{pdfpages}
\usepackage[colorlinks = true,
            linkcolor = blue,
            urlcolor  = blue,
            citecolor = blue,
            anchorcolor = blue]{hyperref}
\usepackage[explicit]{titlesec}
\usepackage{pstricks}
\usepackage[amsmath,thmmarks]{ntheorem} %pakke til at lave sætningsenvorinmets (kan ikke loades sammen med amsthm)
\usepackage{color}
\usepackage{tikz}

%opretter environmets til sætningsstrukturen 
\theorembodyfont{\normalfont}

	
	%sætnings environment	
	\newtheorem{thm}{Sætning}

	\theoremstyle{break}	
	%opgave environment	
	\newtheorem{opg}{Opgave}	

	%Korrolar environment
	\newtheorem{korollar}[thm]{Korollar}	
	
	%Lemma environment	
	\newtheorem{lemma}[thm]{Lemma}
	
	\theoremsymbol{\ensuremath{\circ}}	
	
	%definition environment	
	\newtheorem{definition}[thm]{Definition}
	
	%eksempel environment	
	\newtheorem{eksempel}[thm]{Eksempel}
	
	
	
	%Bevis environment
	\theoremstyle{nonumberplain}
	\theoremheaderfont{%
	\normalfont\itshape}
	\theorembodyfont{\normalfont}
	\theoremsymbol{\ensuremath{\square}}
	\theoremseparator{.}
	
	\newtheorem{proof}{Bevis}
	\newtheorem{los}{Løsning}
	






\setlength\parindent{0pt}

%\titleformat{\section}{\Large\bfseries}{}{0pt}{#1}
%\titleformat{\subsection}{\large\bfseries}{}{0pt}{#1}


%nye komandoer
\newcommand{\mR}{\mathbb{R}}
\newcommand{\mZ}{\mathbb{Z}}
\newcommand{\mN}{\mathbb{N}}
\newcommand{\mQ}{\mathbb{Q}}
\newcommand{\mC}{\mathbb{C}}
\newcommand{\hs}{\hspace{2mm}}
\newcommand{\Hs}{\hspace{4mm}}
\newcommand{\pipe}{\hs | \hs}
\newcommand{\lp}{\left(}
\newcommand{\rp}{\right)}
\newcommand{\vect}[1]{\underline{#1}}
\newcommand{\matr}[1]{\underline{\underline{#1}}}
\newcommand{\cnum}[1]{\raisebox{.5pt}{\textcircled{\raisebox{-.9pt} {#1}}}}




\author{Mikkel B. Goldschmidt\\3r, Nørre Gymnasium}
\title{Laplace lov}
\date{\today}



\begin{document}
\maketitle

\section{Indledning}
Laplaces lov beskriver den kraft, F, der påvirker en elektrisk leder leder med længde, l, hvorigennem der løber en strøm med styrke I, der er anbragt i et magnetfelt med styrke, B, med en vinkel $\Theta$ mellem magnetfeltet og lederen.

Den siger at: 
$$F=B \cdot I \cdot L\cdot \sin (\theta)$$

\section{Teori}
Laplace lov som den vil blive behandlet i denne rapport, er udelukkende en empirisk begrundet lov. 
Dog har den nogle interessante følger i forhold til kraften på elektroner og andre ladede partikler i et magnetfelt, som jeg vil behandle i et perspektiverende afsnit. 

Laplace lov er lidt mere beskrivende en den form jeg beskrev i indledningen. 
Hvis man betragter kraften, strømmen retning og magnetfeltet som vektorer $\overrightarrow{F}$, $\overrightarrow{l}$ og $\overrightarrow{B}$.
Bemærk at magnetfeltet egentlig er et vektorfelt, men at alle relevante vektorer i feltet er lig hinanden da vi arbejder i et homogent magnetfelt. 
Vi kan derfor tillade os at behandle det som en vektor $\overrightarrow{B}$, som er lig en vilkårlig vektor i vektorfeltet. 

Den lov vi vil eftervise er beskrevet med vektorer på følgende måde:
\newcommand{\vek}[1]{\overrightarrow{#1}}

$$\vek{F} = I\cdot (\vek{l}\times \vek{B})$$

Ved at huske at der for krydsproduktet af to vilkårlige vektorer,$\vek{v}$ og $\vek{u}$ med vinkel $\theta$ mellem sig, at 

$|\vek{v} \times \vek{u}|=|\vek{v}|\cdot |\vek{u}| \cdot \sin (\theta)$ ved vi altså at den ovenfor angivne sammenhæng angiver den samme længe af vektor $\vek{F}$ som den første formel, den angiver også bare en længde.

I dette forsøg kommer vi kun til at kigge på et scenarie hvor $\theta=90$. 
Derfor vil kraften altid have samme retning og $\sin (\theta) = 1$ og vi får altså bare en sammenhæng på formen:
$$F = B\cdot I\cdot L$$
som vi vil undersøge.

\pagebreak  
\section{Forsøgsbeskrivelse}
Vi udførte tre forsøg til at bekræfte loven. 
I alle tre forsøg har vi haft en ''Hestesko''-magnet anbragt på en vægt. 
Ned imellem de to poler nede i hesteskoen, satte vi en elektrisk fastsat på en plade der var sat i et stativ.
Gennem denne leder løb der jævnstrøm, der kunne justeres fra en strømforsyning. 
I serieforbindelse mellem strømforsyningen og og lederen var et amperemeter placeret.
På figur \ref{opstilling} kan ses et billede af den beskrevne opstilling.	


\begin{figure}[h]
\center
\includegraphics[scale=0.4]{opstilling}
\caption{Et billede af forsøgsopstillingen for alle tre delforsøg. Billedet er taget af Erik Westergaard fra www.matematikogfysik.dk}
\label{opstilling}
\end{figure}

Vi lavede da tre måleserier hvor vi kiggede på ændringen i $F$, når vi varierede henholdsvis $B,I$ og $L$, mens vi holdt de andre konstante. 
\pagebreak
\section{Data}
I alle tre delforsøg har vi målt en vægt som aflæst på vægten. 
Ud fra denne vægt har vi beregnet en kraft, ud fra at $F = m \cdot g$, hvor $9.82 \frac{N}{kg}$.
\subsection{I,F - måleserie}
Vi målte her vægten $m$ i forhold til strømstyrken gennem lederen $I$:

\begin{table}[h]
\centering
\begin{tabular}{|l|ll|}
\hline
\textbf{$I (A)$} & \textbf{$m(g)$} & \textbf{$F(mN)$} \\ \hline
0                & 165,11          & 1,0802          \\
0,38             & 164,95          & 2,6514          \\
0,41             & 164,94          & 2,7496          \\
0,55             & 164,87          & 3,437           \\
0,66             & 164,83          & 3,8298          \\
0,76             & 164,79          & 4,2226          \\
0,89             & 164,73          & 4,8118          \\
1,09             & 164,65          & 5,5974          \\
1,22             & 164,59          & 6,1866          \\
1,3              & 164,56          & 6,4812          \\
1,48             & 164,48          & 7,2668          \\
1,67             & 164,4           & 8,0524          \\
1,96             & 164,27          & 9,329           \\
2,28             & 164,14          & 10,6056         \\
2,64             & 163,98          & 12,1768         \\
2,99             & 163,84          & 13,5516         \\
3,25             & 163,72          & 14,73           \\
3,36             & 163,67          & 15,221          \\
3,47             & 163,63          & 15,6138         \\
3,78             & 163,49          & 16,9886         \\
3,83             & 163,48          & 17,0868         \\
3,98             & 163,41          & 17,7742         \\
\hline

\end{tabular}
\caption{Delforsøg 1. Snorens længde var $3.2 cm$.}
\end{table}

\pagebreak
\subsection{L, F - Måleserie}
Vi målte her vægten i forhold til længden af lederen. 
Bemærk at vægten har været nulstillet inden målingerne, og vi har derfor kun regnet massedifferenser. 

\begin{table}[h]
\centering

\label{my-label}
\begin{tabular}{|l|ll|}
\hline
\textbf{Længde ($cm$)} & \textbf{Masse($g$)} & \textbf{Kraft ($mN$)} \\ \hline
1,2                    & -0,43               & 4,2226                \\
2,2                    & -0,71               & 6,9722                \\
3,2                    & -0,96               & 9,4272                \\
4,2                    & -1,22               & 11,9804               \\
6,4                    & -1,82               & 17,8724               \\
8,4                    & -2,3                & 22,586               \\
\hline
\end{tabular}
\caption{Delforsøg 2. Strømstyrken var $2.04A$.}
\end{table}

\subsection{B, F - Måleserie}
I dette forsøg har vi varieret antallet af magneter. 
Da det magnetfelt de skaber er ukendt, har vi blot noteret antallet af magneter.

\begin{table}[h]
\centering

\label{my-label}
\begin{tabular}{|l|ll|}
\hline
\textbf{Magneter ($B$)} & \textbf{Masse($m$)} & \textbf{Kraft($mN$)} \\ \hline
1                       & 12,33               & 6,1866               \\
2                       & 24,67               & 12,275               \\
3                       & 37,07               & 17,7742              \\
4                       & 49,57               & 22,2914              \\
5                       & 62,06               & 26,9068              \\
6                       & 74,65               & 30,5402   			\\
\hline          
\end{tabular}

\caption{Delforsøg 3. Strømstyrken var $2.01 A$, og lederens længde var $8.4cm$.}
\end{table}

\section{Databehandling}
Da vi forventer en lineær sammenhæng i alle tre tilfælde, tegner jeg grafer for disse og laver en lineær regression:
\subsection{Delforsøg 1}
\begin{center}
\includegraphics[scale=0.5]{graf1}
\end{center}

\subsection{Delforsøg 2}
\begin{center}
\includegraphics[scale=0.5]{graf2}
\end{center}

\subsection{Delforsøg 3}
\begin{center}
\includegraphics[scale=0.5]{graf3}
\end{center}


\end{document}